\newpage

\section*{\centering{RESUMO}} 

\noindent
A cimentação primária de poços é uma fase crucial das atividades de \textit{upstream} na indústria de petróleo e gás para proteger o revestimento metálico instalado no poço, isolar as formações rochosas permoporosas impedindo a comunicação entre ela, além de promover estabilidade mecânica e impedir efeitos adversos no espaço anular. Em particular, colchões lavadores são aplicados com a finalidade de remover impurezas remanescentes após a perfuração e  facilitar a aderência do cimento à parede porosa. Neste trabalho, analisamos o comportamento hidrodinâmico de um colchão lavador com propriedades físicas equivalentes a de uma emulsão à base de óleo vegetal no espaço anular existente entre o revestimento (\textit{casing}) e a formação rochosa. O escoamento é simulado de maneira simplificada usando a hipótese de fluido Newtoniano, modelado pelas equações de Navier-Stokes para o caso incompressível. Simulações numéricas são executadas via método de elementos finitos a partir da biblioteca FEniCs para configurações bidimensionais representativas de poços submetidos a excentricidades (taxas de \textit{standoff} menores do que 100\%) e erosões. Por fim, comparamos os efeitos de \textit{standoff} e de erosões com modelos homólogos não erodidos e calculamos a eficiência de varrido para caso particular. Os resultados mostram que o perfil de velocidade é alterado quando consideramos malhas cujos contornos assemelham-se a paredes erodidas.

\vspace{1cm}

\noindent{\bf Palavras-chave:}
petróleo e gás,
cimentação primária, 
limpeza de poços, 
dinâmica dos fluidos computacional.