\newpage
\section*{\centering{ABSTRACT}} 

\noindent 
Primary well cementing is a crucial stage of upstream activities in the oil and gas industry to protect the metallic casing inside the well, prevent fluid migration through fractures and avoid adverse effects on the annular space. In particular, flusher fluids are applied to clean post-drilling impurities and facilitate cement's adhesion to the porous wall. In this report, we analyzed the hydrodynamic behavior of a flusher with physical properties equivalent to a vegetable oil-based emulsion moving through the annular space between the casing and the rock formation wall. The incompressible fluid flow was simulated under a Newtonian hypothesis and modeled by the Navier-Stokes equations. Numerical simulations were performed via finite element method from the FEniCs library for two-dimensional configurations representing wells subjected to standoff rates less than 100\% and erosion. Finally, we compared the effects of eccentricity and erosions against homologous non-eroded models and calculated the sweep efficiency for each case. Our findings show that the velocity profile is changed when one consider meshes whose boundaries mimic eroded walls.

\vspace{1cm}

\noindent{\bf Keywords:} 
oil and gas,
primary cementing, 
well cleaning, 
computational fluid dynamics.
