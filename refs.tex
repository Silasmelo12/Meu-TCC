% As referências devem seguir o padrão da ABNT.

%para livro%
%SOBRENOME, Nome; SOBRENOME, Nome; SOBRENOME, Nome. Título: subtítulo (se houver). Edição (se houver). Local: Editora, ano de publicação.

\bibitem{KOEHLER}KOEHLER, Leonardo Pereira. {\bf Projeto de revestimento de popos e suas especificações.} 2018.

\bibitem{Elias} ELIAS, R. N. {\bf Estrutura de dados por arestas para a simulação paralela de escoamentos incompressíveis pelo método estabilizado de elementos finitos.} UFRJ, 2007.

\bibitem{Erik} NELSON, Erik B.; GUILLOT, Dominique. {\bf Well Cementing}. 2ª Edition. 

\bibitem{COTTA} COTTA, Pery. {\bf O petróleo é nosso?}. Guavira, Rio de Janeiro, 3 de outubro de 1953.

\bibitem{White} WHITE, FRANK M. . {\bf Fluid Mechanics}. 7ª edição. University of Rhode Island: McGraw-Hill, 2008.

\bibitem{Logg} LOGG, Anders. {\bf Automated solution of differential equations by the finite element method:} The FEniCS book. 2011.

\bibitem{Langtangen}LANGTANGEN, Hans Petter; LOGG, Anders. {\bf Solving PDEs in Python}: The FEniCS Tutorial Volume I. Center for Biomedical Computing, Simula Research Laboratory and Department of Informatics, University of Oslo. Springer. 2017.

\bibitem{Macondo}COMMISSION, National.  {\bf Macondo: The Gulf Oil Disaster, Chief Counsel's Report}. National Commission. 2011.

\bibitem{Rocha}ROCHA, Luiz; AZEVEDO, Cecilia. {\bf Projetos De Poços De Petróleo}: Geopressões e assentamento de colunas de revestimento. 3ª edição. Interciência, 2019.

%Artigo em um evento
%SOBRENOME, Nome. Título do trabalho apresentado. In: TÍTULO DO EVENTO, nº do evento, ano de realização, local (cidade de realização). Título do documento (anais, resumos, etc). Local: Editora, ano de publicação. Páginas inicial-final.

\bibitem{Curbelo} CURBELO F. D. S.; ARANHA R. M.; MOCHIZUKI V. L.; GUARNICA A. I. C.; FREITAS J. C. O.; SILVA R. K. P. {\bf Lavadores compostos por óleo vegetal, tensoativo e salmoura.} XXI Congresso Brasileiro de Engenharia Química, Fortaleza: Cobeq, 2016.

%Artigo de periódico ou revista
%SOBRENOME, Nome abreviado. Título do artigo. Título da Revista, Local de publicação, número do volume, páginas inicial-final, mês e ano.

\bibitem{Taylor} TAYLOR, C.; Hood, P. {\bf A numerical solution of the Navier-Stokes equations using the finite element technique}, Comp. and Fluids 1 (1973), 73-100.

\bibitem{Cornthwaite} CORNTHWAITE, JOHN P. {\bf Pressure poison method for the incompressible Navier-Stokes equations using galerkin finite elements.} Faculty of Georgia Southern University in Partial Ful llment. 2003.

\bibitem{Campos} CAMPOS, G. {\bf PROCELAB – Procedimentos e métodos de laboratório destinados à cimentação de poços de Petróleo.} 2001

\bibitem{Bourgoyne}BOURGOYNE, Adam T. {\bf Applied drilling engineering.} 1991

\bibitem{SANTOS}SANTOS, Jaíne Lima Dos et al. {\bf Revestimento e cimentação de poços de petróleo.} Anais II CONEPETRO. Campina Grande: Realize, agosto, 2016.

\bibitem{ARANHA} ARANHA, Rayanne Macêdo et al. {\bf OBTENÇÃO DE COLCHÃO LAVADOR A BASE DE TENSOATIVO E ÓLEO VEGETAL PARA REMOÇÃO DE FLUIDO DE PERFURAÇÃO NÃO AQUOSO.} 2015. Disponível em: <https://www.editorarealize.com.br/artigo/visualizar/10365>. Acesso em: 08/11/2021 10:41

\bibitem{VADIM}TIKHONOV, Vadim S. Tikhonov; BUKASHKINA, Olga S.; GANDIKOTA, Raju. {\bf NUMERICAL SIMULATION OF CASING CENTRALIZATION.}. p. 667-675, July, 2014.

\bibitem{Goda} GODA, K. {\bf A multistep technique with implicit difference schemes for calculating two-dimensional or three-dimensional cavity flows}. Journal of Computational Physics, v. 30, n. 1, p. 76 – 95, 1979.

\bibitem{Hanieh}FOROUSHAN, Hanieh K.;LUND, Bjornar; YTREHUS, Jan David;SAASEN, Arild. {\bf Cement Placement: An Overview of Fluid Displacement Techniques and Modelling.} Energies 2021, 14,v. 573. 01/2021.

\bibitem{Monica}NACCACHE, Mônica. Flow displacement in eroded regions inside annular ducts. Springer. Journal of the Brazilian Society of Mechanical Sciences and Engineering, v.420, p. 1-14, 03/2018.

\bibitem{Escudier}ESCUDIER, M.P.; OLIVEIRA, P.J.; PINHO, F.T. {\bf Fully developed laminar flow of purely viscous non-Newtonian
liquids through annuli, including the effects of eccentricity and inner-cylinder rotation}.Elsevier. International Journal of Heat and Fluid Flow, v. 23, p. 52–73, 04/2002.

\bibitem{dissertacao}LIMA, Jaíne. {\bf Revestimento e cimentação de poços de petróleo}: 2006. Dissertação – Física, Universidade Federal de Sergipe, Sergipe, 2006.

\bibitem{CHORIN}CHORIN, A. {\bf Numerical solution of the navier–stokes equations.} 1968

%Referência de monografia, dissertação ou tese
%SOBRENOME, Nome. Título: subtítulo (se houver). Ano de apresentação. Número de folhas ou volumes. Categoria (área de concentração) – Instituição, Local, ano da defesa.

\bibitem{MOLON}MOLON, Fernando. {\bf Simulações numéricas de problemas descritos pelas equações de Navier-Stokes incompressíveis via biblioteca FEniCS.} – Engenharia Mecânica, Universidade Federal do Espírito Santo, Espírito Sant, 2017.

\bibitem{Lupyana}LUPYANA, Samwel Daud. {\bf The Influence of Velocity Profile on Cement Displacement Efficiency}. Norwegian University of Science and Technology, Department of Petroleum Engineering and Applied Geophysics, 2015.

\bibitem{Pordeus}PORDEUS, Roberto {\bf Fenômenos de transporte mecânica dos fluidos:} Considerações e Propriedades dos Fluidos. Universidade Federal Rural do Semi-Árido - UFERSA. 2017.

%Referências de sites
%SOBRENOME, Nome. Título da matéria. Nome do jornal, cidade de publicação (se houver), dia, mês e ano. Seção (caso exista). Disponível em: URL. Acesso em: dia, mês e ano.

\bibitem{Danilo}JANÚNCIO, Danilo. História: 'Petróleo é nosso' leva à criação do monopólio. Folha, 03/10/2003. Disponível em: https://www1.folha.uol.com.br/folha/especial/2003/petrobras50anos/fj0310200303.shtml. Acesso em: 15/05/2022.

\bibitem{}Geuzaine, C. J.-F. Remacle. Gmsh: a three-dimensional finite element mesh generator with built-in pre- and post-processing facilities. International Journal for Numerical Methods in Engineering 79(11), pp. 1309-1331, 2009. Disponível em:  http://gmsh.info. Acesso em 20/05/2022

\bibitem{Max}ALTMAN, Max. {\bf Hoje na História: 1859 - Perfurado o primeiro poço de petróleo nos EUA.} 2020. Disponível em: $<$https://operamundi.uol.com.br/hoje-na-historia/5976/hoje-na-historia-1859-perfurado-o-primeiro-poco-de-petroleo-nos-eua$>$. Acesso em: 8 11 2021.

\bibitem{Petrobras}petrobras. {\bf Batemos sucessivos recordes de produção no pré-sal.} Cidade: Organização, ano. Disponível em: $<$https://petrobras.com.br/fatos-e-dados/batemos-sucessivos-recordes-de-producao-no-pre-sal.htm$>$. Acesso em: 8 11 2021.

%outros
\bibitem{WOLFRAM}WOLFRAM, Stephen. {\bf  Implications for Everyday Systems}. A New Kind of Science. Page 996.

\bibitem{Gmsh} https://gmsh.info/

\bibitem{Fenics} https://fenicsproject.org/

\bibitem{GmshDocuments} https://gmsh.info/doc/texinfo/gmsh.html

