\section{INTRODUÇÃO}

O primeiro poço de petróleo foi perfurado na Pensilvânia, Estados Unidos, em meados do século XIX. Esse poço tinha apenas 23 metros de profundidade quando começou seu ciclo produtivo. Naquela época, o petróleo, até então utilizado como combustível para lamparinas a óleo, gradativamente passou a ser destilado para produzir combustíveis, como o querosene. A descoberta de novas jazidas faria surgir cidades em pleno deserto americano, caracterizando o início da febre do ``ouro negro''. Dali em diante, a tecnologia envolvida na perfuração de poços de petróleo evoluiu consideravelmente e, com o advento da computação, houve grandes melhorias nos processos de extração \cite{Max}.

A história do petróleo no Brasil começa com Eugênio Ferreira de Camargo, quem, em 1892, tentou encontrar petróleo na cidade paulista de Bofete após perfurar um poço de 488 metros. Todavia, achou-se apenas água. Essa foi a primeira tentativa para encontrar petróleo no Brasil. Entretanto, apenas em 1939 descobriu-se a primeira jazida de petróleo na cidade de Salvador, Bahia \cite{Danilo}.

A demanda por petróleo ainda é elevada, visto que é matéria-prima para muitos produtos derivados, tais como plástico, asfalto e gasolina. Com o avanço tecnológico dos processos de recuperação, a indústria de petróleo e gás passou a explorar o recurso em lugares cada vez mais profundos, às vezes ultrapassando os 5000 metros, a exemplo do petróleo retirado do Pré-Sal brasileiro \cite{Petrobras}. Entretanto, essa enorme profundidade leva a desafios operacionais no âmbito da perfuração, completação e estabilidade dos poços.

Durante a atividade de perfuração, injeta-se um fluido por dentro da coluna de perfuração, que avança desde a superfície até à broca. Na broca há um orifício por onde ele é expelido e segue o seu caminho de retorno à superfície passando pelo espaço anular existente entre a coluna de perfuração e a parede do poço. Durante o movimento de retorno, o fluido transporta cascalhos ou fragmentos de rocha que são gerados pela perfuração da formação. Esse fluido é chamado de \textit{fluido de perfuração} \cite{Escudier}.

No processo de perfuração, introduz-se um revestimento metálico com o intuito de proteger e isolar a formação rochosa \cite{KOEHLER}. Em seguida, o espaço anular compreendido entre o revestimento e a formação necessita de preenchimento. Esse preenchimento é realizado por meio do processo de cimentação \cite{SANTOS}. A cimentação é uma das fases mais importantes na  construção de poços de petróleo, a qual tem por função proteger o revestimento metálico instalado no poço e assim impedir o movimento de fluidos através do espaço anular \cite{Bourgoyne}. 

O fluido de perfuração e a pasta de cimento são na maioria das vezes incompatíveis. Por isso, é necessário remover o fluido de perfuração para que a pasta de cimento tenha aderência à formação rochosa. Para tanto, são injetados fluidos intermediários antes da injeção do cimento que são chamados de ``colchões'' (\textit{flushers}), a saber, o ``colchão lavador'' e o ``colchão espaçador''. Esses fluidos têm como função remover completamente o fluido de perfuração da parede do poço \cite{ARANHA}.

Há inúmeros desafios na extração do petróleo. Quanto mais profundo o poço, mais problemas surgem. Um desses problemas é a ocorrência de deslocamento por excentricidade, fenômeno usualmente chamado de \textit{standoff}. O \textit{standoff} ocorre quando o revestimento é inserido no poço e não se conforma perfeitamente centralizado até o fundo do poço, causando má eficiência no processo de limpeza. Consequentemente, a aderência do cimento à formação torna-se comprometida, provocando potenciais problemas de infiltração de fluidos indesejáveis para o interior do poço, além de problemas mecânicos e estruturais, tais como fraturas que podem induzir a demolição do poço \cite{Erik}. 

O presente trabalho analisa o comportamento hidrodinâmico de um colchão lavador com propriedades físicas equivalentes às de uma emulsão a base de óleo vegetal no espaço anular existente entre o revestimento (\textit{casing}) e a formação rochosa. Realizamos simulações de escoamento de maneira simplificada usando a hipótese de fluido newtoniano, modelado pelas equações de Navier-Stokes para o caso incompressível. Os experimentos numéricos são executados via método de elementos finitos a partir da biblioteca FEniCS para configurações bidimensionais representativas de poços submetidos a excentricidades (taxas de \textit{standoff} menores do que 100\%) e erosões. Por fim, os efeitos de \textit{standoff} e de erosões foram comparados com modelos homólogos não erodidos e foi calculada a eficiência de varrido para o caso particular.

