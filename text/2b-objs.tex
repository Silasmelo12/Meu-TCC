\subsection{Objetivos de Pesquisa}

A Matemática Computacional é capaz de propor modelos matemáticos para resolver problemas reais de alta complexidade. Aqui, procuramos dar enfoque à melhoria de processos industriais relacionados à perfuração de poços de petróleo, um tema interdisciplinar que integra diversas áreas de conhecimento, tais como matemática, física, computação e as engenharias química, mecânica e de petróleo. 

Estudar o comportamento do escoamento de colchões lavadores e sua eficiência de limpeza em diferentes configurações de poços traz benefícios para o desenvolvimento científico e tecnológico nacional, além de motivar talentos para a pesquisa. A seguir, destacamos os principais objetivos estabelecidos para a pesquisa.

\subsubsection{Objetivo Geral}

Simular numericamente o escoamento de colchões lavadores caracterizados por propriedades constituintes similares às de óleos biodegradáveis não inflamáveis que se desenvolvem durante a etapa de pré-cimentação de poços de petróleo.

\subsubsection{Objetivos Específicos}

\begin{itemize}
	\item Aplicar o método dos elementos finitos para solucionar numericamente as equações de Navier-Stokes para fluidos incompressíveis;
	\item Implementar códigos computacionais tomando como base a biblioteca FEniCS;
	\item Gerar malhas numéricas para diferentes características geométricas de poço e de espaço anular, simulando efeitos de erosão e excentricidade (\textit{standoff});
	\item Definir um parâmetro para quantificar a eficiência de varredura do escoamento em regime laminar;
	\item Analisar resultados de simulação para configurações de poço bidimensionais;
\end{itemize}

